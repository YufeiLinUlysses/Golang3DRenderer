 \documentclass[a4paper,12pt]{report}

%Packages Used
\usepackage{amssymb,latexsym,amsmath}     % Standard packages
\usepackage{setspace}
\usepackage{sectsty}
\usepackage{titlesec}
\usepackage{hyperref}
\usepackage{bookmark}
\usepackage{graphics,graphicx}
\usepackage{tikz}
\usepackage{mathtools}
\usepackage{graphicx}
\usepackage{esvect}

\DeclarePairedDelimiter\abs{\lvert}{\rvert}%
\DeclarePairedDelimiter\norm{\lVert}{\rVert}%


\bookmarksetup{
  numbered,
  open
}
\renewcommand*{\thesection}{\arabic{section}}
\onehalfspacing

%Margins
\addtolength{\textwidth}{1.0in}
\addtolength{\textheight}{1.00in}
\addtolength{\evensidemargin}{-0.75in}
\addtolength{\oddsidemargin}{-0.75in}
\addtolength{\topmargin}{-.50in}

%%%%%%%%%%%%%%%%%%%%%%%%%%%%%% 
% Theorem/Proof Environments %
%%%%%%%%%%%%%%%%%%%%%%%%%%%%%%
\newtheorem{theorem}{Theorem}
\newenvironment{proof}{\noindent{\bf Proof:}}{$\hfill \Box$ \vspace{10pt}}
\sectionfont{\fontsize{12}{15}\selectfont}
\titlespacing*{\section}{0.5pt}{0.25\baselineskip}{0.25\baselineskip}

\begin{document}
\noindent
Yufei Lin

\noindent
Advance Computer Graphic Final

\noindent
Nov \(21^{th}\) 2019

\begin{center}
\textbf{Torus Primitive}
\end{center}

Torus is one of the most important shape in our life. We have bagels, doughnuts and many other things. In order to have a better reconstruction of our daily life scenes in a 3D renderer, the best way to introduce torus as a single primitive instead of a combination of triangles or any other shapes. 

\begin{figure}[h]
\centering
\includegraphics[scale=1]{./Pic/Torus1.png}
\caption{Torus Perspective View}
\end{figure}

\noindent
\textbf{I. Defining a Torus}

There are three circles in a torus $C_1, C_2, C_3$ as shown in the graph below. $C_2$ is how a torus is defined, because the center of the tube is uniformly lying on it and by utilizing it we can reduce our equation for the torus.  Then, we have the radius of $C_2$ as $R$ and the raidus of the tube as $r$. Therefore, we know that from this graph:
\(x^2+y^2=R^2\).
\begin{figure}[h]
\centering
\includegraphics[scale=0.4]{./Pic/Torus2.png}
\caption{Torus Top View}
\end{figure}

Then, we just need to obtain the torus from $x,y$ and $z$. Based on observation, we can see the cross section of a torus as the following graph, and could interpret that the size of the tube should be a function like the following: $(g(x,y, R))^2+z^2=r^2$, where $g(x,y,R)$ defines where the center of a tube should lie on the middle circle.
\begin{figure}[h]
\centering
\includegraphics[scale=0.75]{./Pic/Torus3.png}
\caption{Torus Cross Section}
\end{figure}

From the previous equation we know that $x^2 +y^2 = R^2$, and we could just sample the point 
$\begin{pmatrix}
(R-r)sin\pi\\
(R-r)cos\pi\\
0
\end{pmatrix}$. This means that we should have $g(x,y,R)=\sqrt{x^2+y^2}-R$ to achieve the length of the tube. Therefore, the equation of a torus is
\begin{equation}
(\sqrt{x^2+y^2}-R)^2+z^2=r^2
\end{equation}

\noindent
\textbf{II. Finding the Intersection With a Ray}

In the ray tracer, we see each light and our camera as ray. And we define them as the following:
\begin{equation}
\overrightarrow{v} = 
\overrightarrow{o}+\overrightarrow{d}
=
\begin{pmatrix}
x_{origin}\\
y_{origin}\\
z_{origin}
\end{pmatrix} + t \cdot{
\begin{pmatrix}
x_{direction}\\
y_{direction}\\
z_{direction}
\end{pmatrix}}
\end{equation}

Thus, we have 
\begin{equation}
\begin{cases}
x=x_{origin}+t\cdot{x_{direction}}\\
y=y_{origin}+t\cdot{y_{direction}}\\
z=z_{origin}+t\cdot{z_{direction}}
\end{cases}
\end{equation}

\pagebreak
Therefore, when we define an intersection between the ray and the torus, we would have the following five situations, with 0 to 4 intersections on the torus:
\begin{figure}[h]
\centering
\includegraphics[scale=0.75]{./Pic/Torus4.png}
\caption{Torus Intersections on the Top View}
\end{figure}

Thus, we first rewrite (1) as the following: 
\begin{equation}
(x^2+y^2+z^2+R^2-r^2)^2=4R^2(x^2+y^2)
\end{equation}

Then, we find the intersection by replacing $x,y$ and $z$ in (4) with the equations we have in (3). Thus we have:
\begin{align*}
&((x_{origin}+t\cdot{x_{direction}})^2\\
+&(y_{origin}+t\cdot{y_{direction}})^2\\
+&(z_{origin}+t\cdot{z_{direction}})^2+R^2-r^2)^2\\
=&[(x_{direction}^2+y_{direction}^2+z_{direction}^2)t^2\\
+&2(x_{origin}x_{direction}+y_{origin}y_{direction}+z_{origin}z_{direction})t\\
+&(x_{origin}^2+y_{origin}^2+z_{origin}^2+R^2-r^2)]^2\\
=&4R^2(((x_{origin}+t\cdot{x_{direction}})^2+(y_{origin}+t\cdot{y_{direction}})^2)\\
=&4R^2((x_{direction}^2+y_{direction}^2)t^2\\
+&2(x_{origin}x_{direction}+y_{origin}y_{direction})t+(x_{origin}^2+y_{origin}^2))
\end{align*}

Since all the coefficient are constant and all of $x_{origin},y_{origin},z_{origin},x_{direction},y_{direction},z_{direction},R,r$ are constants, then we can replace all of them with certain letters for simplification. Also, we will have all these constant calculated at the beginning of our \textit{IntersectionWithRay} function. Thus, we have the following:

\[\begin{cases}
a=4R^2(x_{direction}^2+y_{direction}^2)\\
b=8R^2(x_{origin}x_{direction}+y_{origin}y_{direction})\\
c=4R^2(x_{origin}^2+y_{origin}^2)\\
d=(x_{direction}^2+y_{direction}^2+z_{direction}^2)=|\overrightarrow{d}|^2\\
e=2(x_{origin}x_{direction}+y_{origin}y_{direction}+z_{origin}z_{direction})=\overrightarrow{o}\cdot{\overrightarrow{d}}\\
f=(x_{origin}^2+y_{origin}^2+z_{origin}^2+R^2-r^2)=|\overrightarrow{o}|^2+R^2-r^2\\
\end{cases}\]

Then by applying all these constants back, we have: 
\begin{align*}
&a\cdot{t^2}+b\cdot{t}+c\\
=&(e\cdot{t^2}+f\cdot{t}+g)^2\\
=&e^2t^4+2eft^3+(2ge+f^2)t^2+2gft+g^2
\end{align*}

Therefore, we have obtained the following quartic equation for $t$:
\[e^2t^4+2eft^3+(2ge+f^2-a)t^2+(2gf-b)t+g^2-c=0\]
\end{document}